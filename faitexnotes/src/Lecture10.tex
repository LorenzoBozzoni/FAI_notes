\section{Constraint Satisfaction Problems: Formulation and Solving}
Let's start from the search trees. We have seen that the search trees are composed by nodes. Each node corresponds to a path from the initial state to a state in the state space. Each state of the space state can correspond to multiple nodes, when a state can be reached, from the initial state, following multiple paths. 

But, considering the implementation we have made of the node, we can easily see that the state of a node is written in a very generic manner. \textbf{ We have not specified any structure for it so the states can be anything. Plain search algorithms only compare states using = and $\neq$.}\\

What we want to do now is to specialize the representation of the states in a search problem. Now it rises a new tradeoff:
\[
    \text{expressiveness} \leftrightarrow \text{efficiency}    
\]
When the structure of the state is fixed, more efficient solving techniques could be developed. In constraint satisfaction problems, the state has a \textbf{factored representation}, namely it is represented as a set of pairs variable - value. 

\subsection{8-queens problem}
We consider once again this problem in which your goal is to position 8 queens on a chessboard so that no queen can attack any other queen. A naive formulation is:
\begin{itemize}
    \item Initial state: empty board
    \item Actions(): add a queen to an empty cell
    \item Result(): \dots
    \item Goal test: 8 queens that do not attack each other
    \item Step cost: 1
\end{itemize}
A more efficient formulation is:
\begin{itemize}
    \item Place a queen on a cell
    \item Do not consider attacked cells
    \item Coun the number of cells that are not attacked in each row and column
    \item Place a queen in the cell corresponding to the row and column with the smallest number
    \item Do not consider attacked cells
    \item Repeat \dots
\end{itemize}
\begin{multicols}{4}
    \begin{center}
        \includegraphics[scale=0.3]{../images/10/QP2.png}
        \includegraphics[scale=0.3]{../images/10/QP6.png}
        \includegraphics[scale=0.3]{../images/10/QP3.png}
        \includegraphics[scale=0.3]{../images/10/QP7.png}
        \includegraphics[scale=0.3]{../images/10/QP4.png}
        \includegraphics[scale=0.3]{../images/10/QP8.png}
        \includegraphics[scale=0.3]{../images/10/QP5.png}
        \includegraphics[scale=0.3]{../images/10/QP9.png}
    \end{center}
\end{multicols}

\textbf{What can an efficient formulation provide?} Empowering function Actions() and goal test, "propagate the constraints" in order to limit the actions applicable in a state and identifying failures as soon as possible (failure test).

\subsection{Constraint Satisfaction Problems (CSPs)}
They are defined as follows:
\begin{itemize}
    \item Set of \textbf{variables}: $\{X_1, \dots, X_n\}$
    \item \textbf{Domains}: $D_1, \dots, D_n$ of possible values for each variable
    \item Set of \textbf{Constraints}: $\{C_1, \dots, C_p\}$. Each constraint involves a subset of variables and specifies the valid (or invalid) combinations of their values
\end{itemize}
The goal is to assign a value to each variable in such a way that all the constraints are satisfied. \textbf{Solution}: \textbf{complete} (all variables are assigned a value) and \textbf{consistent} (all constraints are
satisfied) assignments.

\noindent\textbf{Example: Map Coloring}
\begin{multicols}{2}
    \begin{center}
        \includegraphics[scale=0.3]{../images/10/MapColoring.png}
    \end{center}
    \begin{itemize}
        \item 7 variables: $\{WA, NT, Q, NSW, V, SA, T\}$
        \item Each variable has the same domain: $D_i = \{red, green, blue\}$
        \item Variables corresponding to adjacent regions cannot have the same color: $WA \neq NT, WA \neq SA, NT \neq SA, NT \neq Q, SA \neq Q, SA \neq NSW, SA \neq V, Q \neq NSW, NSW \neq V$
    \end{itemize}
\end{multicols}

\noindent\textbf{Example: 8-queens}
\begin{itemize}
    \item 8 variables $X_i, i = 1, 2, \dots, 8$, each one representing the row of the queen on the column $i$ (or vice versa)
    \item Each variable has the same domain $D_i = \{1, 2, \dots, 8\}$
    \item Constraints are of the form:
    \begin{itemize}
        \item no two queens on the same row: $X_i \neq X_j, i,j = 1,2, \dots, 8, i \neq j$
        \item no two queens on the same diagonal: $|X_i - X_j| \neq |i - j|, i,j = 1,2, \dots, 8, i \neq j$
    \end{itemize}
\end{itemize}

\noindent\textbf{Example: Zebra puzzle}
There are five houses.
\begin{multicols}{2}
    \begin{itemize}
        \item The Englishman lives in the red house
        \item The Spaniard owns the dog
        \item Coffee is drunk in the green house
        \item The Ukrainian drinks tea
        \item The green house is immediately to the right of the ivory house
        \item The Old Gold smoker owns snails
        \item Kools are smoked in the yellow house
        \item Milk is drunk in the middle house
        \item The Norwegian lives in the first house
        \item The man who smokes Chesterfields lives in the house next to the man with the fox
        \item Kools are smoked in the house next to the house where the horse is kept
        \item The Lucky Strike smoker drinks orange juice
        \item The Japanese smokes Parliaments
        \item The Norwegian lives next to the blue house
    \end{itemize}
\end{multicols}
Who drinks water? Who owns the zebra?\\
\begin{itemize}
    \item 5 variables, the "houses"
    \item Each variable is a vector of 5 values: nationality, color, beverages, cigarettes, animals
    \begin{itemize}
        \item $N_i = \{\text{Englishman, Spanish, Ukrainian, Japanese, Norwegian}\}$
        \item $C_i = \{\text{red, green, ivory, yellow, blue}\}$
        \item $B_i = \{\text{coffee, tea, milk, orange juice, water}\}$
        \item $S_i = \{\text{Old Gold, Kools, Chesterfields, Lucky Strike, Parliaments}\}$
        \item $A_i = \{\text{dog, snalis, fox, horse, zebra}\}$
    \end{itemize}
    \item Constraints:
    \begin{itemize}
        \item The nationality of a house is different from the nationality of the other houses $\rightarrow$ if $X_i = [v, \dots]$ then $X_j = [v', \dots]$ with $v \neq v'$ for all $j \neq i$
        \item The color of a house is different from the color of the other houses
        \item \dots
        \item The Englishman lives in the red house $\rightarrow$ if $X_i = [\text{Englishman}, \dots]$ then $X_i = [\text{Englishman}, \text{red}, \dots]$
        \item The Norwegian lives in the first house $\rightarrow$ $X_1 = [\text{Norwegian}, \dots]$
        \item \dots
    \end{itemize}
\end{itemize}

\subsection{Assignments}
Consider $n$ variables: $X_1, X_2, \dots, X_n$. 

Definition of \textbf{consistent assignment}: $\{X_{i1} \leftarrow v_{i1}, X_{i2} \leftarrow v_{i2}, \dots, X_{ik} \leftarrow v_{ik}\}$ with $0 \leq k \leq n$ such that values $v_{i1}, v_{i2}, \dots, v_{ik}$ satisfy all constraints involving variables $X_{i1}, X_{i2}, \dots, X_{ik}$.\\

Definition of \textbf{complete assignment}: $\{X_{i1} \leftarrow v_{i1}, X_{i2} \leftarrow v_{i2}, \dots, X_{ik} \leftarrow v_{ik}\}$ with $k = n$.\\

If the domains of the variables have size $d$, there are $O(d^n)$ complete assignments.


\subsection{Binary constraints}
We usually consider binary constraints, namely with constraints involving two variables. Any generic CSP can be transformed in a CSP with only binary constraints. The structure of a CSP with binary constraints is represented by a constraint graph.
\begin{center}
    \includegraphics[scale = 0.4]{../images/10/Australia.png}
\end{center}

\textbf{Example - Binary constraints}
Oridinal problem: $X = \{x,y,z\}$, $D = \{D_y = \{1,2\}, D_x = \{3,4\}, D_z = \{5,6\}\}$, $C = \{C_1 = [x + y = z]\}$. Constraint $C_1$ is not binary $\rightarrow$ binarization.
\textbf{Goal}: formulate a new problem equivalent to the original one but with only binary constraints. Hint: introduce a new variable with a complex domain\dots 

Introduce a new variable $u$ whose domain is the Cartesian product of the domains of the variables involved in $C_1$: $D_u = \{(1,3,5),(1,3,6),(1,4,5),(1,4,6),(2,3,5),(2,3,6),(2,4,5),(2,4,6)\}$. Reduce $D_u$ keeping only the valid combinations according to $C_1$: $D_u = \{(1,4,5),(2,3,5),(2,4,6)\}$ ("$u$ encapsulates $C_1$"). Add consistency constraints: $u[1] = x, u[2] = y, u[3] = z$.

Implementation of CSP:
\begin{Verbatim}[frame = single, numbers = left, fontsize=\small]
class CSP(search.Problem):
    """This class describes finite-domain Constraint Satisfaction Problems.
    A CSP is specified by the following inputs:
        variables A list of variables; each is atomic (e.g. int or string).
        domains A dict of {var:[possible_value, ...]} entries.
        neighbors A dict of {var:[var,...]} that for each variable lists
            the other variables that participate in constraints.
        constraints A function f(A, a, B, b) that returns true if neighbors
            A, B satisfy the constraint when they have values A=a, B=b

    def MapColoringCSP(colors, neighbors):
        """Make a CSP for the problem of coloring a map with different colors
        for any two adjacent regions. Arguments are a list of colors, and a
        dict of {region: [neighbor,...]} entries. This dict may also be
        specified as a string of the form defined by parse_neighbors."""
        if isinstance(neighbors, str):
            neighbors = parse_neighbors(neighbors)
        return CSP(list(neighbors.keys()), UniversalDict(colors), neighbors, different_values_constraint)

    Australia_csp = MapColoringCSP(list('RGB'), """SA: WA NT Q NSW V; NT: WA Q; NSW: Q V; T: """)
\end{Verbatim}

\subsection{CSPs as search problems}
\begin{itemize}
    \item States: consistent assignments
    \item Initial state: empty assignment $\{\}$, namely $k = 0$
    \item Actions($\{X_{i1} \leftarrow v_{i1}, X_{i2} \leftarrow v_{i2}, \dots, X_{ik} \leftarrow v_{ik}\}$): all possible consistent assignment for $X_{ik + 1}$
    \item Result($\{X_{i1} \leftarrow v_{i1}, X_{i2} \leftarrow v_{i2}, \dots, X_{ik} \leftarrow v_{ik}\}$, $X_{ik + 1} \leftarrow v_{ik + 1}$) = $\{X_{i1} \leftarrow v_{i1}, X_{i2} \leftarrow v_{i2}, \dots, X_{ik} \leftarrow v_{ik}, X_{ik + 1} \leftarrow v_{ik + 1}\}$
    \item Goal test: $k = n$
    \item Step cost: unitary
\end{itemize}
\begin{center}
    \includegraphics[scale = 0.3]{../images/10/CspAsSearchProbl.png}
\end{center}
$r = n - k$ variables that are not yet assigned with $d$ values $\rightarrow$ branching factor $r \times d$.

\subsubsection{Commutativity}
The order in which values are assigned to the variables has no impact on the possibility of reaching consistent and complete assignments. Expand a node N by choosing one variable $X$ not included in the assignment associated to N
and by assigning it a value from its domain (the new assignment must be consistent) $\rightarrow$ reduce branching factor.\\

\noindent\textbf{Example - Commutativity}
Consider 4 variables $X_1, X_2, X_3, X_4$. Consider a node N with (consistent) assignment $\{X_1 \leftarrow v_1, X_2 \leftarrow v_2\}$. Consider, for instance, variable $X_4$ with domain $\{v_{4,1}, v_{4,2}, v_{4,3}\}$. Successors of N are all the nodes with consistent assignments: 
\begin{itemize}
    \item $\{X_1 \leftarrow v_1, X_2 \leftarrow v_2, X_4 \leftarrow v_{4,1}\}$
    \item $\{X_1 \leftarrow v_1, X_2 \leftarrow v_2, X_4 \leftarrow v_{4,2}\}$
    \item $\{X_1 \leftarrow v_1, X_2 \leftarrow v_2, X_4 \leftarrow v_{4,3}\}$
\end{itemize} 
\begin{center}
    \includegraphics[scale = 0.3]{../images/10/Commutativity.png}
\end{center}
The order in which values are assigned to the variables has no impact on the possibility of reaching consistent and complete assignments. 
\begin{enumerate}
    \item Expand a node N by choosing one variable $X$ not included in the assignment associated to N and by assigning it a value from its domain (the new assignment must be consistent) $\rightarrow$ reduce branching factor
    \item The path for reaching a node is not relevant and \textbf{all solutions are at depth $n$ $\rightarrow$ depth-first search with backtracking}
\end{enumerate}

\subsubsection{Backatracking: consider a single value}
Example with 3 variables: ($x_1, x_2, x_3$) and domains: $D_1 = \{v_{11}, v_{12}, \dots\}$, $D_2 = \{v_{21}, \dots\}$, $D_3 = \{v_{31}, v_{32}\}$.
\begin{multicols}{2}
    \begin{center}
        \includegraphics[scale = 0.25]{../images/10/B1.png}
        \includegraphics[scale = 0.25]{../images/10/B2.png}
        \includegraphics[scale = 0.25]{../images/10/B3.png}
        \includegraphics[scale = 0.25]{../images/10/B4.png}
        \includegraphics[scale = 0.25]{../images/10/B5.png}
        \includegraphics[scale = 0.25]{../images/10/B6.png}
        \includegraphics[scale = 0.25]{../images/10/B7.png}
        \includegraphics[scale = 0.25]{../images/10/B8.png}
        \includegraphics[scale = 0.25]{../images/10/B9.png}
        \includegraphics[scale = 0.25]{../images/10/B10.png}
        \includegraphics[scale = 0.25]{../images/10/B11.png}
        \includegraphics[scale = 0.25]{../images/10/B12.png}
    \end{center}    
\end{multicols}
Here is the backtracking algorithm:
\begin{Verbatim}[frame = single, numbers = left, fontsize=\small]
CSP-BACKTRACKING(A)
    IF assignment A is complete, THEN RETURN A
    X ← choose a variable not in A
    D ← choose an order of values in the domain of X
    FOR EACH v in D DO
        add (X ← v) to A
        IF A is consistent THEN
            result ← CSP-BACKTRACKING(A)
        IF result != failure, THEN RETURN result
            remove (X ← v) from A
    RETURN failure
    
    
def backtracking_search(csp,select_unassigned_variable=first_unassigned_variable,
order_domain_values=unordered_domain_values, inference=no_inference):
    def backtrack(assignment):
        if len(assignment) == len(csp.variables):
            return assignment
        var = select_unassigned_variable(assignment, csp)
        for value in order_domain_values(var, assignment, csp):
            if 0 == csp.nconflicts(var, value, assignment):
                csp.assign(var, value, assignment)
                removals = csp.suppose(var, value)
                if inference(csp, var, value, assignment, removals):
                    result = backtrack(assignment)
                    if result is not None:
                        return result
                csp.restore(removals)
        csp.unassign(var, assignment)
        return None

    result = backtrack({})
    assert result is None or csp.goal_test(result)
    return result
\end{Verbatim}